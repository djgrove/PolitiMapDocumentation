\documentclass{article}
\usepackage{varioref}
\usepackage[hidelinks]{hyperref}
\usepackage{cleveref}
\usepackage{tabu}
\usepackage{booktabs}
\newcommand{\story}[3]{\item As a #1, I want to be able to #2 so I can #3.}
\newcommand{\requirement}[1]{The system shall #1.}

\title{User Stories, Use Cases\\
  {\large and}\\
  Functional Requirements}
\author{Devon Grove}

\begin{document}
\maketitle
\tableofcontents
\section{User Stories}
\begin{enumerate}
\story{San Luis Obispo community member}{view local political news through a centralized medium}
	{remain objectively informed about local politics}\label{agenda}
\story{Cal Poly student}{view policy positions of local politicians}
	{become encouraged to participate in local politics and make informed decisions in local elections}
\story{San Luis Obispo City Council Member}{contact my constituents in a more intimate way than email}
	{increase transparency and my understanding of community needs and perspectives}
\end{enumerate}
\section{Use Cases}
\begin{tabu}{rX}
  \toprule
  Use Case ID: & 5\\
  Use Case Name: & Display Policy Positions\\
  Created By: & Devon Grove\\
  Last Updated By: & Devon Grove\\
  Date Created: & 2016-10-07\\
  Date Last Updated: & 2016-10-07\\
  Actors: & The PolitiMap App\\
  Description: & View policy summary of given local politician on active issues\\
  Preconditions: &
  \begin{enumerate}
  \item The user's location is currently supported by the app
  \item The user has chosen the city council member to evaluate
  \item Push
  \item The user has selected "Policy Positions" from menu options
  \end{enumerate} \\
  Postconditions: &
  \begin{enumerate}
  \item The app displays a summarized list of policy positions on active local legislation, and
  	lists a "position not available" message if position is indeterminate
  \end{enumerate} \\
  Normal Flow: &
  \begin{enumerate}
  \item The user opens the app
  \item The user specifies their location, or a cached location is processed
  \item The user selects "Politicians" from the lower UI menu
  \item The user selects their chosen politician from the "Your Representatives" list
  \item The user navigates to "Policy Positions" using the UI menu
  \end{enumerate} \\
  Priority: & Medium\\
  Special Requirements: & None\\
  Assumptions: & Policy positions will be available publicly through government or council member's website\\
  \bottomrule
\end{tabu}
\section{Functional Requirements}
\begin{tabu}{lX}
  \toprule
  FR Requirement & Description\\
  \midrule
  FR-1 & \requirement{load data in the JSON format specified in
    \vref{json-format}}\\
  FR-2 & \requirement{fetch the data using AJAX or other method of web request}\\
  FR-3 & \requirement{only load JSON data for the locations selected
    by the user}\\
  FR-4 & \requirement{data on politicians must be verifiable and completely accurate}\\
  \bottomrule
\end{tabu}
\section{Non-Functional Requirements}
\begin{tabu}{lX}
  \toprule
  NFR Requirement & Description\\
  \midrule
  NFR-1 & \requirement{provide menu UI on politician page for ease of navigation}\\
  NFR-2 & \requirement{display bulleted list of policy positions on active legislation}\\
  \bottomrule
\end{tabu}

\section{JSON Data Formatting}
\label{json-format}
This has yet to be determined. At this time, more consultation with the team is necessary to determine the exact data constraints and schema requirements for the project.
Until it has been determined if major, generic policy points will be utilized or active local issues, schema cannot be defined. Whether "issue" data will be static or dynamic
will affect the scope of this requirement.
\end{document}