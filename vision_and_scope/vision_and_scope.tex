\documentclass{article}
\usepackage{calc}

\usepackage{tabu}
\usepackage{booktabs}

\usepackage{ragged2e}

\usepackage{adjustbox}

\usepackage{biblatex}
\addbibresource{bibliography.bib}

\title{Vision and Scope Document\\\large for\\\Large PolitiMap}
\author{PolitiMap Team}
\begin{document}
\maketitle
\tableofcontents
\section{Revision History}
\section{Business Requirements}
Ultimate value of the product: Encourage political involvement in
local government. This benefits the users by enabling them to better
take advantage of the republic in which they live. This can also
benefit activist groups by enabling them to get more eyes on the
causes they care about.

\subsection{Background}
Voter turnout is a problem for the United States as a whole, with a
study from Pew Research stating that the US ranks “31st among the 35
countries in the Organization for Economic Cooperation and
Development, most of whose members are highly developed, democratic
states.”  \cite{pew} Voter turnout for local elections in large cities
is even worse, with a 2014 study from Governing showing that turnout
in Chicago, New York, Philadelphia, and LA ranged from 20\% to 42\% in
the most recent election shown in the study. \cite{governing} Turnout
in San Luis Obispo county for the most recent general election in 2014
was 58.42\% of registered voters. \cite{slocounty}

\subsection{Business Opportunity}
Given that 62,434 registered voters did not vote in the most recent
general election, there is an opportunity to significantly increase
voter turnout in San Luis Obispo county. Further, judging from the low
turnout for elections, it seems reasonable that people who do not vote
likely are not following the progress of local bills. PolitiMap will
seek to increase the awareness of local voters regarding local bills.

\subsection{Business Objectives and Success Criteria}
\begin{tabu}{lX}
  \toprule
  Business Objectives & Metric\\
  \midrule
  \textbf{BO-1} Have active users & \textbf{SC-1} 100 calls made by
  voters to their representatives in local government.\\
  \bottomrule
\end{tabu}

\subsection{Customer or Market Needs}
TBD

\subsection{Business Risks}
\begin{tabu}{lXX}
  \toprule
  Risk Level & Description & Mitigation \\
  \midrule
  Occurred &
  \textbf{RI-1} Competitor exists &
  Gain clear
  understanding of competitor's positioning and either provide a
  better product or a different product\\

  High &
  \textbf{RI-2} Server hosting may become expensive after our
  free year on AWS &
  Make plans to shut down system after one year or to
  have long-term funding \\

  Low &
  \textbf{RI-3} A server breakin may allow information to be
  changed &
  Proper server security \\
  Low & \textbf{RI-4} A server breakin may allow misuse of the server
& Proper server security \\
  \bottomrule
\end{tabu}

\section{Vision of Solution}
\subsection{Vision Statement}
PolitiMap seeks to increase the involvement of ordinary citizens in
local government.

\subsection{Major Features}
\begin{tabu}{lX}
  \toprule
  Feature & Description\\
  \midrule
  \textbf{FE-1} & Display of local, state, and federal bills\\
  \textbf{FE-2} & Easy way to call representative\\
  \bottomrule
\end{tabu}

\subsection{Assumptions and Dependencies}
\begin{tabu}{lX}
  \toprule
  Assumption    & Description \\
  \midrule
  \textbf{AS-1} & More involvement with local politics is beneficial \\

  \textbf{AS-2} & Voters lack information in order to be more involved
  with local politics \\
  \textbf{AS-3} & Voter involvement with local politics is
  limited\footnote{This seems likely to be true, given voter turnout
    numbers, but may not be.} \\
  \bottomrule
\end{tabu}
\vspace{.3in}
\noindent\begin{tabu}{lX}
  \toprule
  Dependency    & Description \\
  \midrule
  \textbf{DE-1} &
  Ability to get data on bills in a machine-parsable format (even if
  that is initially plaintext with custom code to parse it) \\
  \bottomrule
\end{tabu}

\section{Scope and Limitations}
\subsection{Scope of Initial Release}
Initially, the application will only provide a way of viewing current
bills in the San Luis Obispo City Council.

\subsection{Scope of Subsequent Releases}
Following a successful initial release, the next releases would be
expected to add the ability to display current bills in the San Luis
Obispo County Supervisors, followed by the California State
Legislature and the US Congress. Push notifications would be added to
notify people when a new bill comes up.

Further expansion to cover more geographical locations would be possible.

\subsection{Limitations and Exclusions}
We do not plan to implement any functionality regarding candidates in
elections in the first versions of the application. Such a feature
might be possible in the long term, but not within the first few
releases.


\section{Business Context}
\subsection{Stakeholder Profiles}
\begin{adjustbox}{center,minipage=[c]{\paperwidth - 1in}}
  \begin{tabu}{lX[L]X[L]X[L]X[L]}
    \toprule
    Stakeholder & Major Value & Attitudes & Major Interests & Constraints\\
    \midrule
    Local Voters&
    Makes activity in local government easier&
    Likely indifference, given apparent current attitude toward local government&
    Ways in which policies and elections will impact their everyday lives&
    Ease of use: information must be available quickly\\

    Local activist groups&
    Make bills affecting their causes more widely known&
    Depending on the influence of the application, hopefully interest&
    Passing and electing officials that represent their points of activism&
    May or may not be direct users; competition\\

    Local government officials&
    Will receive more calls from voters if the application succeeds.&
    Currently unknown.&
    Properly and accurately serving their constituents&
    Not direct users of the application\\
    \bottomrule
  \end{tabu}
\end{adjustbox}

\subsection{Project Priorities}
TBD

\subsection{Operating Environment}
This project will be run as an iPhone mobile application aimed at
users across the United States. Since our initial target audience is
limited to a domestic one, the geographic distribution won’t be such
that servers at many unique locations will be necessary. Our server(s)
will be running a database with MySQL as the DBMS, and utilize PHP
code on server-side to query the database and communicate with the
iPhone app using requests and JSON data. Using Rest API calls to a
backend database as the data management system is common practice in
mobile application development.

At this stage, we have not explored revenue streams for our business
model besides advertising. At present, allowing user access for free
somewhat reduces the risk incurred by a major service outage on our
app since the degree of pressure involved when users pay for access to
the application is significant. PolitiMap is not a tool we expect to
be used more than once or twice a day by customers, and an outage of
up to a full 24 hours should not be a major burden.

Our MySQL server should be secured to prevent injection and
manipulation of our data, but none of the data hosted on it is
expected to be sensitive.

\begin{appendix}
\printbibliography
\end{appendix}
\end{document}
