\subsection{Overview}
See our architeceture diagram in \vref{architecture}.
\begin{figure}
\begin{center}
\includegraphics[width=.8\linewidth]{PolitiMapAppFlow}
\end{center}
\caption{\label{architecture}Architecture Diagram}
\end{figure}
\subsubsection{Frameworks, Tools, and Libraries Utilized}
PolitiMap is run as an iPhone Mobile Application written in Swift. Due
to our initial constraints, hosting servers were not necessary. JSON data
is hosted using an Amazon S3 filespace bucket. To make RESTful API calls
from the app, PolitiMap uses the Alamofire Swift library. JSON processing
is simplified using the SwiftyJSON to handle implicit data types inherent
with JSON data.
\subsubsection{Data Flow}
% would be good if Andrew discussed some of the work that went into parsing
% data here, if it can be done concisely
% also a good idea to have a diagram here
Application data is split between an Amazon S3 service and local storage.
Information pertinent to political data is housed on Amazon, while user
preferences are cached locally on the device. This eliminates the need for
a full server running a Database Management System (such as MySQL). Remote
data is stored in manually-created JSON files. When a user attempts to view
information about a region within PolitiMap, the app uses Alamofire
to make a request to the JSON files on the web and returns them. The source
code handles the implicit JSON data using native SwiftyJSON functions to
identify data types. Once it is typed, the data is loaded into the 
